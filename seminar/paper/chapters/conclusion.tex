\section{Conclusion}
\label{sec:Conclusion}

In this paper, we presented gitc, a Cloud9 extension that allows for a better programming experience when working with git. 
Based on our sample implementation of \textit{Tetris}, we evaluated possible Cloud9 features that would considerably improve the IDE and could be reasonably implemented in the time range of one semester.
Gitc provides experienced git users with the functionality and usability they already know from desktop tools. 
By tracking and displaying code changes they receive immediate feedback on their actions. 
Creating this "immediate connection" was one of our initial goals of this seminar. 
Gitc also makes git newcomers learn the git process in a much quicker and more comfortable way. Buttons that are placed right where code changes happen lead the user through the git stage-commit-push workflow and give them an intuitive understanding of what a chunk is.

Technically, although we managed to achieve most of our goals, there is still room for potential improvements. 
Most of all performance of realtime updates of code changes needs to be tested in a real world environment. 
Since we could only run and develop in a local environment, optimizations may become necessary. 
We also think of additional features, that we think would yet further improve the experience of our tools. 
Section ~\ref{sec:future_work} gives an overview of potential features.

Our implemented features address the issues with the current Cloud9 version. 
We've been contacted by the Cloud9 team and work together on getting \emph{gitc} into the main Cloud9 IDE\footnote{\url{http://c9.io}}.
\section{Cloud9 by Example}
\label{sec:Motivation}

%\begin{itemize}
%	what c9 can do
%	Tetris
%	how was our development process (team of 3 (2 know git already), cloud9 dashboard, setting up c9 %project (github or bitbucket or without any of them))
%	where do we had problems,
%	our pros and cons for cloud9
%\end{itemize}

%more detailed description of git console
%include Cloud9 Screenshot
%include Tetris Screenshot
After registering for an account on the Cloud9 website or logging in through GitHub or BitBucket, the user is led to the Dashboard, where he can create new or maintain existing projects as well as access those shared with him by other members. Cloud9 supports many languages which amongst others are NodeJS, Javascript, HTML, CSS, Python, Ruby, PHP, C/C++, C\#, Java, and Scala. While the first two of the mentioned are fully supported in terms of running and debugging other languages are only provided with syntax highlighting. Once a workspace is created - either from scratch or by cloning from a URL from GitHub or BitBucket - and project is launched, the actual editing environment appears. The user interface is arranged into three parts. The top is covered by a menu bar, followed by a navigation pane on the left side of the screen and the editing window on the right. At the bottom of the screen is an expandable git command shell, which accepts the entire range of Git commands. The menu includes the mandatory items known from any desktop application such as file or edit operations, text search or view settings. The navigation pane involves various views, accessible through tab icons. The most important is the project explorer, which displays all project files in a tree view. Here, project files and folders can be edited, deleted or newly created. Files that cannot be created or edited in Cloud9 such as images can be dragged and dropped straight from a local hard disk. Other views include a Run \& Debug operations, Server Deployment or text editor settings. The latter serves the actual coding purposes and supports multiple tabs. Under the hood runs an instance of the Ace Editor\footnote{http://ace.ajax.org/}, originally known as Bespin initially developed by Mozilla Labs, and provides the user with many of the mandatory text editing features such as syntax highlighting, grouped code indention, and line numbers. The new release also allows code completion, which has previously only been supported for some commands, function or variable names that already appeared in one document. A more detailed view on the Ace editor, its backend components and functionality is provided in Chapter ~\needcite. 

\subsection{Developing Tetris}
There is not much to say about Tetris, one of the world's most famous computer games. Single out of four squares composed objects, that fall from the top border of the rectangular playing field, need to be arranged to form complete, horizontal lines at the bottom of the field. Our Javascript based implementation offers two players to battle each other with the use of a key shared keyboard. Players score by dropping objects and generating unbroken lines. Accomplishing more than one line at once will add the corresponding number of lines to the opponent player's field. The first player running out of space for new spawning objects loses. We duplicated the shapes of the original game and included six different levels. Each level upgrade results in a higher score for each dropped object and cleared line as well as in a higher falling speed of the objects.

Apart from creating the elements of the graphical user interface, which required the use of image processing software, we were able to create the game entirely through the use of a web browser. 
Even though it understands the entire range of Git commands, suggestions are only made for a small selection. As one of our team members has not been familiar with Git a more sophisticated implementation of this feature would have resulted in a much more comfortable learning process. Keyword or paragraph highlighting, e.g. added/deleted chunks, is also missing. 

\subsection{Cloud9 Alternatives}
%list of other tools --> dart, Jens' list
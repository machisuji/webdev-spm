\section{gitc extension}
\label{sec:Extension}
In this section we describe how a cloud9 user can work with our extension and we show how the developing workflow is improved by it.
Further we compare our extension to other tools introduced in section~\needcite{related work}.
Finally we will explain how we intregrated our extension to cloud9 and how we implemented it.

\subsection{Gitc User Interface}
\label{sec:gitc_ui}
\paragraph{The editor adjustments} of our extension will promt the cloud9 user immediately with visiual feedback of source code changes.
That means while typing within the editor annotations will appear next to the left grutter line as can be seen in figure~\ref{fig:editor}.

The changes show the staged and unstaged changes of the git repository respectively.
We choose to display those both types of changes as colored annotations whereas the already staged changes are more transparent.
In the upper screenshot of the cloud9 editor are only unstaged changes displayed.
The lower screenshot shows that the state of the git repository is changed.
Some of the changes are staged and at~\circnum{2} and~\circnum{3} are some (new) unstaged changes.
So in the line marked by~\circnum{2} there are both staged and unstaged changes visualized.
The colors green, blue and red are used to visualize added~\circnum{3}, changed~\circnum{2} and deleted~\circnum{1} lines respectively.

Furthermore the user will get tooltips hovering over one annotation.
In this way deleted lines or the old content of changed lines will be displayed to the user.
We do not provide buttons to stage, unstage or discard the changes because there is already an extension which allows developers to go back in history (see figure~\ref{fig:history}).
As for the staging and unstaging we want to have an overview over all current changes and thus rather use the diff view than search in single source code files for changes.

\begin{figure}
   \centering
   \includegraphics[width=0.9\textwidth]{images/extension_tooltip_comparison.png}
   \caption{Editor adjustments of our cloud9 extension gitc.}
   \label{fig:editor}
\end{figure}

\paragraph{Using our diff view} the cloud9 developer has now a view to explore \texttt{git diff} visually.
By clicking on the pane button \circnum{1} in figure~\needcite or simply using the keyboard shortcut \texttt{strg + g} or \texttt{cmd + g} ....

\begin{figure}
   \centering
   \includegraphics[width=0.9\textwidth]{images/extension_unstage.png}
   \caption{The diff view of our cloud9 extension gitc.}
   \label{fig:diff_view}
\end{figure}

comparison to earlier editing way, mention history extension (and why we have choosen not to implement stage etc buttons here)
comparison to earlier diffing, committing

comparison to related work

\subsection{Implementation}
In the following subsections we will describe how we implemented our extension.
First we will have a closer look at how we intregrated our extension to cloud9 and how we execute git commands.
Then we will explain how we implemented the adjustments in the cloud9 editor and the new diff view.

\subsubsection{Integration to cloud9}
%Stephi

\subsubsection{Editor Adjustments}
%ask for git diffs when creating document
%creating and storing annotations, unsaved changes
%using ace api: markers
%registered events: document creation, tab switch, document change

Once a document is opened or becomes active for the first time in the editor, its current workspace version is checked against the repository version. For that purpose Gitc registered with the \texttt{documentCreate} and \texttt{tabSwitch} events emitted by the ACE editor. Sending the command \texttt{git diff [--cached] -U0} to the client side Gitc Interface returns the differences between the two file versions, split into the various chunks. Staged and unstaged changes for each document file are transformed into objects that we refer to as \texttt{annotations} and maintained in an array. Annotations are constantly updated as long as the user edits the document. Once saved, the document is checked against the repository version again and the annotations array is refilled. ACE provides \texttt{documentChange} and \texttt{saveFile} events in order to keep track of change and save operations introduced to the document file. In order to highlight code changes according to annotations, ACE stores various states related to a document in an \texttt{EditSession}, which provides a function \texttt{addMarker}. A marker is a highlighted range of text or text background over one or more lines in the editing window. In our implementation we make use of markers in order to produce the narrow bars at the left border of the window as already described in \ref{sec:gitc_ui}. Markers are created and updated every time a document is opened, saved or edited. Subscribing to the ACE event \texttt{mouseMove} enables hovering elements in the code window, which we use to create our tooltips that relate to code changes as in figure~\ref{fig:editor}.

\subsubsection{Diff View}
%Markus

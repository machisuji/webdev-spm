\section{Introducing Web-based Software Development}
\label{sec:Introduction}

%- why do we want to do web-dev?
%	- everything moves to the cloud, cloud is one of the biggest key trend throughout the last years
%	- social media, social engineering is another
%	- web turns into an application platform
%	- multiple workplaces: office, home --> code everywhere
%	- how gitc supports the idea of getting instant feedback while developing
%	--> mention Bret Victor talk http://vimeo.com/36579366
%	- pros and cons in general
%		++ collaboration
%		++ application runs in the web, not localhost --> real world conditions for testing
%		-- requires internet connection
	
%list some environments
%	- Cloud9
%	- Jens list (dart, lively, ...)
%	- brackets
%	- collide
	
%include some stats: #web-apps vs "normal" websites, 	amount of user generated content, etc
Since the rise of Web 2.0 technologies, the Web has changed from a mainly static network containing textual information
to an application platform that serves multiple kinds of purposes.
Applications encompass everything from games to communication services, to social media platforms for private users
as well as collaboration tools for business users.
Not only the types of content have changed from textual to any conceivable media format, but also the way it is generated.
Consumers have become creators as people write blogs, share their lives with their friends online and upload videos.
The amount of user generated content has massively increased. In addition, everything moves to the cloud,
providing access to data anywhere, anytime.
However, developing these highly complex applications has long stayed a task mainly to be carried out offline,
discarding many of the benefits connecting through the web has to offer.
The already almost ubiquitous existence of online code repositories and code sharing platforms like
GitHub\footnote{\url{https://github.com/}} let programmers backup their code and connect with others.
However, the process of coding itself involves working locally - most often alone - and the use of multiple tools.

Fortunately, the release of web-based development environments such as Lively Kernel~\ref{Ingalls2008LKS}
has marked a big step in improving this process.
It lets team members store their code in the cloud and therefore makes local workspace copies obsolete.
No matter whether they work from their private notebooks or office desktop computers, their code will always be up to date,
taking care of the trend of the recent trend of multiple devices that one carries today.
As already introduced by other web applications, collaboration features such as in Lively Kernel push social coding
to a higher level, encouraging team development techniques like pair programming.
Most obvious, a reliable internet connection is always needed, which, however, ensures the IDE can also act as a runtime environment at the same time.
Thus, a web application can be tested in a real world setting with no extra test server needed to be set up anymore.
The idea of keeping everything in one browser window to get a more direct and quicker feedback on code changes is
what Bret Victor refers to as an ``immediate connection'' in his talk ``Inventing on Principle'' given at
CUSEC 2012~\ref{victor2012inventing} in Montreal.

In this paper, we present Gitc, a Cloud9 extension for visual realtime Git support which we developed as part of the
summer term 2012 seminar ``Web-based Development Environments'' at Hasso-Plattner-Institute in Potsdam.
Cloud9 is an online IDE written in NodeJS. Released in March 2011 it fully supports NodeJS and Javascript development
while providing the programmer with syntax highlighting for many other programming languages.
For the purpose of diving into the IDE we developed a Tetris clone.


Chapter \ref{sec:Motivation} gives an overview of the features of Cloud9 and how it makes use of the
previously mentioned advantages of a web-based development process.
Based on our experiences with the IDE we point out possible extensions next to Gitc, that we think would improve
the Cloud9 experience, followed by an overview of related work regarding Git support in other IDEs.
In Chapter \ref{sec:Extension}, we provide technical insight into our implementation and outline the important
components to deal with when extending Cloud9.
Finally, we encourage ideas for future work regarding Cloud9 and Gitc in Chapter \ref{sec:Future_Work} and
evaluate the results of our work in Chapter \ref{sec:Conclusion}.
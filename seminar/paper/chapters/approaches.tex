\section{Extending Cloud9}
\label{sec:Approaches}
During the development of our sample application \emph{Tetris} we figured three major issues out that troubled our development work flow.
We briefly address each of them in this chapter.

\subsection{Collaboration}
As mentioned in~\needcite[somewhere in motivation] we implemented our sample application in a team of three people.
We expected to use cloud9 as a collaboration tool.
At the moment we started our project there was no possibilty to add team members to a project on a cloud9 dashboard.
The collaboration in cloud9 was restricted to collaboration with the aid of a git~\needcite or mercurial~\needcite repository.
Thus using a repository as we were used to beforehand was the only way of having collaboration.

Actually we wanted to recognize the doings of other team member during development, e.g.:
\begin{itemize}
	\item Is another team member working on the same project as well?
	\item At what file is the team member working?
	\item At what exactly does the team member is working at?
\end{itemize}
Further we wanted to be albe to communicate with a team member.
For this all wasn't supported by cloud9 this was one extension we planned to implement.

\subsection{IDE-based Issue-Tracking}
Not only during the development of our sample application also during our every day development we wish to be supported by automatic issue tracking.
Thus we want to extend cloud9 to handle automatic issue tracking.
For this extension two different aspects were important for us.
On the one hand we wanted to know how long we've worked on which file.
So this extension would have to track the focus time of files and, if possible, in a more fine-grained level what methods or the like were edited in the file.
On the other hand we wanted automatic issue creation.
As for that we wanted to recognize if the developer annotated any code with e.g. \texttt{TODO}, \texttt{FIXME} or \texttt{BUG} and automatically create issues.
Therefore this extension would include an issue tracker for cloud9.
If the project is linked to a github~\needcite or bitbucket~\needcite repository the repective issue tracking api of either of them should be used to create corresponding issues there as well.

\subsection{Proper git integration}
As we stated in~\needcite[somewhere in motivation] two of us already knew git and developed several projects using git.


\begin{itemize}
	\item our 3 approaches, and what we have choosen
	\item mention version control systems in general, git
\end{itemize}


Collaboration was to be released in about 3 weeks
IDE-based Issue-Tracking
	Measure time and assign to tickets
	Automatically create tickets while coding
Proper git integration in editor / IDE
	Work without console
	Visualize changes
	(Branches, History, Tags, …)


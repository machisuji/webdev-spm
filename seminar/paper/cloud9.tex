\documentclass{llncs}

\usepackage{listings}
\usepackage{url}
\usepackage{graphicx} %png
\usepackage{subfigure} %placing pictures side-by-side
\usepackage{pifont}	%ding
\usepackage[usenames,dvipsnames]{color}	%text color
\usepackage{amssymb, amsmath}
\usepackage{ifthen}
\usepackage{marginnote}

%define JavaScript listing:
\definecolor{lightgray}{rgb}{.9,.9,.9}
\definecolor{darkgray}{rgb}{.4,.4,.4}
\definecolor{purple}{rgb}{0.65, 0.12, 0.82}

\lstdefinelanguage{JavaScript}{
  keywords={typeof, new, true, false, catch, function, return, null, catch, switch, var, if, in, while, do, else, case, break},
  keywordstyle=\color{blue}\bfseries,
  ndkeywords={class, export, boolean, throw, implements, import, this},
  ndkeywordstyle=\color{darkgray}\bfseries,
  identifierstyle=\color{black},
  sensitive=false,
  comment=[l]{//},
  morecomment=[s]{/*}{*/},
  commentstyle=\color{purple}\ttfamily,
  stringstyle=\color{red}\ttfamily,
  morestring=[b]',
  morestring=[b]"
}
\lstset{
   language=JavaScript,
   backgroundcolor=\color{lightgray},
   extendedchars=true,
   basicstyle=\footnotesize\ttfamily,
   showstringspaces=false,
   showspaces=false,
   numbers=left,
   numberstyle=\footnotesize,
   numbersep=9pt,
   tabsize=2,
   breaklines=true,
   showtabs=false,
   captionpos=b
}

% cirnum command to print numbers in circles within text
\newcounter{dingdistance}
\setcounter{dingdistance}{191}
\newcommand{\circnum}[1]{%
  \addtocounter{dingdistance}{#1}%
  \ding{\value{dingdistance}}%
  \setcounter{dingdistance}{191}}

% \todo command
\newcommand{\todo}[1][EMPTY]{%
\null\marginnote[$\rightarrow$]{$\leftarrow$}%
\ifthenelse{\equal{#1}{EMPTY}}%
{[\textsc{TODO}]}%
{[\textsc{TODO:} #1]}%
}
% \needcite command
\newcommand{\needcite}[1][EMPTY]{%
\null\marginnote[$\rightarrow$]{$\leftarrow$}%
\ifthenelse{\equal{#1}{EMPTY}}%
{[\textsc{citation needed}]}%
{[\textsc{citation needed from} #1]}%
}

% packed lists
\newenvironment{packed_enumerate}{
\begin{enumerate}
  \setlength{\itemsep}{1pt}
  \setlength{\parskip}{0pt}
  \setlength{\parsep}{0pt}
}{\end{enumerate}}
\newenvironment{packed_itemize}{
\begin{itemize}
  \setlength{\itemsep}{1pt}
  \setlength{\parskip}{0pt}
  \setlength{\parsep}{0pt}
}{\end{itemize}}
\newenvironment{packed_description}{
\begin{description}
  \setlength{\itemsep}{2pt}
  \setlength{\parskip}{0pt}
  \setlength{\parsep}{0pt}
}{\end{description}}

\begin{document}
\frontmatter
\pagestyle{headings}

\title{Cloud9 - gitc extension}
\author{Markus Kahl \and Stephanie Platz \and Patrick Schilf \\~ \small{advisors: Jens Lincke \and Prof. Dr. Robert Hirschfeld} }
\subtitle{Seminar Web-based Software Development Environments}
\institute{Hasso Plattner Institut \\ University of Potsdam, Germany}
\date{2011-02-18}

\maketitle

\begin{abstract}
\label{abstract} % markus
A very nice abstract.
\end{abstract}

\section{Introducing Web-based Software Development}
\label{sec:Introduction}

\begin{itemize}
	\item why do we want to do web-dev?
	\item mention Bret Victor talk http://vimeo.com/36579366 ?
	\item pros and cons in generall
\end{itemize}

\subsection{Environments}
\begin{itemize}
	\item list some environments, especially c9 ;)
	\item Jens list (dart, lively, ...)
	\item brackets
	\item collide
\end{itemize} %patrick
\section{Cloud9 by Example}
\label{sec:Motivation}

\begin{itemize}
	\item Tetris
	\item how was our development process, where do we had problems	
	\item mention version control systems in general, git
	\item our pros and cons for cloud9
	\item our 3 approaches, and what we have choosen
\end{itemize} %patrick
\section{Extending Cloud9}
\label{sec:Approaches}
During the development of our sample application \emph{Tetris} we figured three major issues out that troubled our development work flow.

\subsection{Collaboration}
As mentioned in \needcite[somewhere in motivation] 

\begin{itemize}
	\item our 3 approaches, and what we have choosen
	\item mention version control systems in general, git
\end{itemize}


Collaboration
	See what other team members do in real-time
	But: will be released in ~3 weeks
IDE-based Issue-Tracking
	Measure time and assign to tickets
	Automatically create tickets while coding
Proper git integration in editor / IDE
	Work without console
	Visualize changes
	(Branches, History, Tags, …)

 %stephi
\section{Related Work}
\label{sec:Related_Work}

As far as we know there is currently no other project aiming to better integrate git into Cloud9.
There are, however, several offline IDEs and stand-alone tools we took as inspiration for our
Cloud9 plugin \emph{gitc}.
Two of them we personally use are:
\begin{itemize}
	\item IntelliJ IDEA\footnote{\url{http://www.jetbrains.com/idea/}}
	\item gitx\footnote{\url{http://gitx.frim.nl/}}
\end{itemize}
In the following sections we will show which features from those tools we intended to integrate into Cloud9.

\subsection{IntelliJ IDEA}
\label{subsec:IntelliJ_IDEA}

IntelliJ IDEA is a very rich Java IDE with plugins for various other languages such as Ruby, PHP and Scala.
The one git feature we set out to integrate into Cloud9 is the live diff inside the code editor.
This means that while you type the IDE shows you what changed and how as compared to the git index.

\begin{figure}
	\centering
	\includegraphics[width=0.9\textwidth]{images/idea-git.png}
	\caption{IntelliJ IDEA git integration}
	\label{fig:idea}
\end{figure}

In Figure ~\ref{fig:idea} you can see a screenshot of IntelliJ IDEA's code editor including the live diff.
At \circnum{1} there is a little gray triangle which marks deleted code.
You can see the deleted code by clicking on it.
Changed lines are marked via a blue strip on the editor's gutter (\circnum{2}).
By clicking that strip the original code is displayed.
Finally, as seen at \circnum{3}, added code is marked by a green strip.

\subsection{gitx}
\label{subsec:gitx}

Bundled with git comes a graphic tool called \emph{gitk} with which you can review local changes
and the history of your git repository. There are several similar tools that offer additional features.
One of those tools is \emph{gitx} for MacOS and one of the additional features is the ability to partially apply and revert
changes, while usually IDEs such as IntelliJ IDEA only allow you to commit files as a whole.

\begin{figure}
	\centering
	\includegraphics[width=0.9\textwidth]{images/gitx-history.png}
	\caption{gitx history view}
	\label{fig:gitx-history}
\end{figure}

Figure ~\ref{fig:gitx-history} shows gitx's history view where you can see the list of all commits.
Below the list of commits the diff for the selected commit is displayed.
Currently Cloud9 only provides a history of local changes independently from git. Another feature \emph{gitc} is supposed to add
is this git history view. Although it hasn't been implemented, yet.
\vspace{11 pt}

\begin{figure}
	\centering
	\includegraphics[width=0.9\textwidth]{images/gitx-commit.png}
	\caption{gitx commit view}
	\label{fig:gitx-commit}
\end{figure}

You can see gitx's commit view in figure ~\ref{fig:gitx-commit}. It shows the diff for local changes grouped into several so called hunks by git.
It's also possible to commit those separately by using the command line (\emph{git add --patch}), though gitx offers a more
convenient UI.
As you can see at \circnum{1} there are stage and discard buttons at every hunk.
Through this you can either add separate hunks to the next commit or revert them.
Staged changes will go from \circnum{2} to \circnum{3} and can be unstaged again separately as well. %markus
\section{gitc extension}
\label{sec:Extension}
In this section we describe how a cloud9 user can work with our extension and we show how the developing workflow is improved by it.
Further we compare our extension to other tools introduced in section~\needcite{related work}.
Finally we will explain how we intregrated our extension to cloud9 and how we implemented it.

\subsection{Description}
\paragraph{The editor adjustments} of our extension will promt the cloud9 user immediately with visiual feedback of source code changes.
That means while typing in the editor annotations will appear next to the left grutter line as can be seen in figure~\needcite ~\circnum{1}.

The changes show the staged and unstaged changes of the git repository.
We choose to display both type of changes as colored annotations whereas the already staged changes are more transparent. \todo[Vgl. left right anntotation circnum]
The color \textcolor{LimeGreen}{green} is used to visualize added, \textcolor{ProcessBlue}{blue} for changed and \textcolor{Red}{red} for deleted lines.

\begin{figure}
   \centering
   \includegraphics[width=0.9\textwidth]{images/extension_tooltip_comparison.png}
   \caption{Caption}
   \label{fig:editor}
\end{figure}

comparison to earlier editing way, mention history extension (and why we have choosen not to implement stage etc buttons here)

\paragraph{Using our diff view} the cloud9 developer can know ....
By clicking on the pane button \circnum{1} in figure~\needcite or simply using the keyboard shortcut \texttt{strg + g} or \texttt{cmd + g} ....

comparison to earlier diffing, committing

comparison to related work

\subsection{Implementation}
In the following subsections we will describe how we implemented our extension.
First we will have a closer look at how we intregrated our extension to cloud9 and how we execute git commands.
Then we will explain how we implemented the adjustments in the cloud9 editor and the new diff view.

\subsubsection{Integration to cloud9}
%Stephi

\subsubsection{Editor Adjustments}
%Patrick

\subsubsection{Diff View}
%Markus
 %stephi: description, new workflow, ansonsten jeder seins
\section{Future Work}
\label{sec:Future_Work}

While we are quite satisfied with the features we managed to implement in the given time,
there is more to be done to make the git integration perfect.
Most importantly the \emph{history view}, but also:
\begin{itemize}
	\item switching branches
	\item checking out specific tags/commits
\end{itemize}
An optional but potentially very useful feature would be \emph{offline} git support.

\subsection{History View}
\label{sec:history-view}

For completion gitc should also feature the history view shown in figure ~\ref{fig:gitx-history}.
This would not necessarily have to look just as in gitx.

\subsection{Branches and checkout out specific tags and commits}
\label{sec:branches}

Currently \emph{gitc} does not support branches. Creating branches as well as checking out branches,
commits etc. still requires the use of the Cloud9 console. It would be desirable to be able to do all
this over the \emph{gitc} view.

\subsection{Offline git support}
\label{sec:offline-git-support}

There is one major restriction while working with git through Cloud9. It does not work offline since everything goes through the Cloud9 server-side component.
Usually this would not be a problem, however it is one of the key points of git that you can work with it (committing etc.)
while offline.
It would be very useful if you could still do at least some things while offline.
Perhaps one could realize that through Dan Lucraft's git implementation git.js\footnote{https://github.com/danlucraft/git.js} in pure Javascript which we could possibly run on the client side to do local commits for example.
 %markus
\section{Conclusion}
\label{sec:Conclusion}

Conclude the paper here. %patrick, all

\bibliographystyle{plain}
\bibliography{cloud9-Refs}

\end{document}

%\begin{figure}
%	\centering
%	\includegraphics[width=\textwidth]{images/picture.png}
%	\caption{A nice Caption}
%	\label{fig:Caption}
%\end{figure}
